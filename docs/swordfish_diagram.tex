% EPC flow charts
% Author: Fabian Schuh
\documentclass{minimal}

\usepackage{pgf}
\usepackage{tikz}
\usepackage{amsmath}
%%%<
\usepackage{verbatim}
\usepackage[active,tightpage]{preview}
\PreviewEnvironment{tikzpicture}
\setlength\PreviewBorder{5pt}%
%%%>

\begin{comment}
:Title:  EPC flow charts
:Grid: 1x2


\end{comment}
\usepackage[utf8]{inputenc}
\usetikzlibrary{arrows,automata}
\usetikzlibrary{positioning}


\tikzset{
    state/.style={
           rectangle,
           rounded corners,
           draw=black, very thick,
           minimum height=2em,
           inner sep=2pt,
           text centered,
           },
}

\begin{document}

\begin{tikzpicture}[->,>=stealth']

 % Position of QUERY 
 % Use previously defined 'state' as layout (see above)
 % use tabular for content to get columns/rows
 % parbox to limit width of the listing
 \node[state,
  fill=white!50!yellow
   ] (MODEL) 
 {\begin{tabular}{l}
  \textbf{Model}\\
      $\Phi(\Omega|\theta)$: Model\\
      $\Phi(\Omega)_{BG}$: Background\\
      $\Sigma(\Omega, \Omega')_{ij}$: Covariance\\
      $\mathcal{E}(\Omega)$: Exposure
 \end{tabular}};
  
 % State: ACK with different content
 \node[state,    	% layout (defined above)
%  text width=3cm, 	% max text width
  %yshift=0cm, 		% move 2cm in y
  right of=MODEL, 	% Position is to the right of QUERY
  node distance=5.0cm, 	% distance to QUERY
  fill=white!50!yellow,
  anchor=center] (INFO) 	% posistion relative to the center of the 'box'
 {%
 \begin{tabular}{l} 	% content
   \parbox{4.7cm}{
   \textbf{Fisher Information Matrix}}
   \\[1mm]
  $\quad\qquad\qquad\mathcal{I}_{ij}(\theta)$
 \end{tabular}
 };
 
 % STATE EPC
 \node[state,
  below of=INFO,
  node distance=2cm,
  xshift = 6cm,
  anchor=center] (GEO) 
 {%
 \begin{tabular}{l}
   \parbox{4cm}{\textbf{Information Geometry}}\\
   $\qquad\qquad g_{ij} = \mathcal{I}_{ij}$
 \end{tabular}
 };

 \node[state, below of=GEO, node distance = 2cm] (CONTOUR)
 {
 \begin{tabular}{l}
   \parbox{4cm}{\textbf{Confidence contours}}\\
   \parbox{4.5cm}{$\simeq$ constant geodesic distance contours}
 \end{tabular}
 };

 \node[state, below right of=INFO,
%  right of=ACK,
  node distance=5cm,
  xshift = -3cm
%  anchor=center
] (FLUX) 
 {%
 \begin{tabular}{l}
   \parbox{4cm}{\textbf{   Information Flux}}\\
 $\mathcal{I}(\theta)_{ij} = \int dt\int d\Omega \frac{d\mathcal{E}(\Omega)}{dt}\mathcal{F}(\Omega)_{ij}$
 \end{tabular}
 };

 \node[state, below of=FLUX, node distance=1.5cm] (EXP)
 {%
   \textbf{Experimental design}
 };

 \node[state, below left of=INFO,
%  right of=ACK,
  node distance=5cm,
  xshift = -2cm
%  anchor=center
] (COUNTS) 
 {%
 \begin{tabular}{l}
   \parbox{4.3cm}{\textbf{Effective counts method}}\\
   $\quad\mathcal{I}_{ij}(\theta) \to \left(s_i(\theta), b_i(\theta)\right)$
 \end{tabular}
 };

 \node[state, below of=COUNTS,
%  right of=ACK,
  node distance=1.5cm,
  xshift = +1.8cm
%  anchor=left
] (LIMITS) 
 {
   \textbf{Exclusion limits}
 };

 \node[state, below of=COUNTS,
%  right of=ACK,
  node distance=1.5cm,
  xshift = -1.8cm
%  anchor=center
] (REACH) 
 {
   \textbf{Discovery reach}
 };

% \node[state, below of=COUNTS,
%%  right of=ACK,
%  node distance=2.5cm,
%%  anchor=center
%] (LIKE) 
% {%
%   \textbf{Model likelihood}
% };



 % draw the paths and and print some Text below/above the graph
% \path (QUERY) edge[bend left=20] node[anchor=south,above]{$SC_n=0$}
%                                    node[anchor=north,below]{$RN_{16}$} (ACK)
% (QUERY)     	edge[bend left=0] node[anchor=south,above]{$SC_n\neq 0$} (QUERYREP)
% (ACK)       	edge                                                     (EPC)
% (EPC)       	edge[bend left]                                          (QUERYREP)
% (QUERYREP)  	edge[loop below]    node[anchor=north,below]{$SC_n\neq 0$} (QUERYREP)
% (QUERYREP)  	edge                node[anchor=left,right]{$SC_n = 0$} (ACK);
\path (MODEL) edge (INFO);
\path (INFO) edge (GEO);
\path (INFO) edge[bend left=0]  (FLUX);
\path (INFO) edge[bend right=0] (COUNTS);
\path (COUNTS) edge (REACH);
\path (COUNTS) edge (LIMITS);
%\path (COUNTS) edge (LIKE);
\path (GEO) edge (CONTOUR);
\path (FLUX) edge (EXP);

\end{tikzpicture}


\end{document}
